\documentclass[a4j,12pt]{gradthesis_utf8}
%\usepackage[dviout]{graphicx}
\usepackage{graphicx}
%
%%% ドラフトモード(図表は,図表のみのページになる)
%\draftmode
%
%%% 2ページ目に英語の題目をいれる
\engtitle
%
%%% 2ページ目に英語の所属をいれる
\engaffil
%
%%% 2ページ目に英語の著者名をいれる
\engauthor
%
%%% ヘッダ(章番号と章タイトル)を入れる
\usehead
%
\jtitle{TCP並列接続を用いたプログレッシブダウンロード\\における順序制御方式の実装} % 和文題目
%
\etitle{Implementation of sequence control method in progressive download using TCP parallel connection} % 英文題目
%
\jaffil{広島市立大学 情報科学部 情報工学科}
\eaffil{Department of Computer and Network Engineering\\
Faculty of Information Sciences\\
Hiroshima City University}
%
\jauthor{1420180 \quad 平城 光雄} % 和文著者名
\eauthor{1420180 \quad Mitsuo Heijo} % 英文著者名
\supervisor{舟坂 淳一}  % 指導教官名
%
%
\jabst{ % 和文梗概 
\hspace*{0.5em}概要
} %

\eabst{ % 英文梗概
\hspace*{1em}gaiyo}

\begin{document} 
\maketitle %とびらの出力

%%%%%%%%%%%%%%%%%%%%%%%%%%%%%%%%%%%%%%%%%%%%%%%%%%%%%%%%%%%%%%%%%%%%%%%%%%%%%
% 第1章
%%%%%%%%%%%%%%%%%%%%%%%%%%%%%%%%%%%%%%%%%%%%%%%%%%%%%%%%%%%%%%%%%%%%%%%%%%%%%
\chapter{はじめに}\label{sec:sec1}
%%% abst %%%
はじめに

%%%%%%%%%%%%%%%%%%%%%%%%%%%%%%%%%%%%%%%%%%%%%%%%%%%%%%%%%%%%%%%%%%%%%%%%%%%%%
% 第2章
%%%%%%%%%%%%%%%%%%%%%%%%%%%%%%%%%%%%%%%%%%%%%%%%%%%%%%%%%%%%%%%%%%%%%%%%%%%%%
\chapter{関連研究}\label{sec:sec2}
本章ではまず,動画配信方式の一つであるプログレッシブダウンロードについて述べる.

\section{プログレッシブダウンロード方式}
プログレッシブダウンロード

%%%%%%%%%%%%%%%%%%%%%%%%%%%%%%%%%%%%%%%%%%%%%%%%%%%
%%%%%%%%%%%%%%%%%%%%%%%%%%%%%%%%%%%%%%%%%%%%%%%%%%%
 \section{複数経路を用いた通信方式}
 複数経路

%%%%%%%%%%%%%%%%%%%%%%%%%%%%%%%%%%%%%%%%%%%%%%%%%%%%%%%%%%%%%%%%%%
 \section{複数のTCP接続を用いた通信方式}
 複数のTCP接続
 %%%%%%%%%%%%%
 
\chapter{提案方式}\label{sec:sec3}
\section{遅延要求方式}
\subsection{固定遅延方式}

\subsection{差分計測を用いた遅延予測方式}

\subsection{接続使用回数比を用いた遅延予測方式}

\section{重複再要求方式}
\subsection{バッファ内非有効ブロック数依存方式}
\subsection{非有効ブロック受信回数依存方式}

\chapter{実装評価}\label{sec:sec4}

\section{提案方式の評価}
\subsection{テストベッドでの評価}

\subsection{実ネットワークでの評価}

\chapter{動画配信サーバーへの適用例}
\section{プロキシでの実装}
 
%%%%%%%%%%%%%%%%%%%%%%%%%%%%%%%%%%%%%%%%%%%%%%%%%%%%%%%%%%%%%%%%%%%%%%%%%%%%
% 第X章
%%%%%%%%%%%%%%%%%%%%%%%%%%%%%%%%%%%%%%%%%%%%%%%%%%%%%%%%%%%%%%%%%%%%%%%%%%%%%
\chapter{今後の課題}\label{sec:sec7}
\hspace*{0.5em}今後の課題として, 以下が挙げられる.
\begin{itemize}
	\item 実際のユーザー体験を考慮した評価
	\item その他
\end{itemize}
\clearpage
%
%%%%%%%%%%%%%%%%%%%%%%%%%%%%%%%%%%%%%%%%%%%%%%%%%%%%%%%%%%%%%%%%%%%%%%%%%%%%%
% 謝辞
%%%%%%%%%%%%%%%%%%%%%%%%%%%%%%%%%%%%%%%%%%%%%%%%%%%%%%%%%%%%%%%%%%%%%%%%%%%%%
\begin{acknowledgment}
 本研究の機会を与えて頂き,多くの御指導,および御助言を賜わりました
舟坂 淳一 准教授に深甚なる謝意を表します.また,その他多くの御助言を頂きま
した諸氏に心より感謝致します.
\end{acknowledgment}
%%%%%%%%%%%%%%%%%%%%%%%%%%%%%%%%%%%%%%%%%%%%%%%%%%%%%%%%%%%%%%%%%%%%%%%%%%%%%
% 参考文献
%%%%%%%%%%%%%%%%%%%%%%%%%%%%%%%%%%%%%%%%%%%%%%%%%%%%%%%%%%%%%%%%%%%%%%%%%%%%%
\begin {thebibliography}{20} 
\bibitem{test} test

\end {thebibliography}

\end{document}
